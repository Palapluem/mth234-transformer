\documentclass[12pt,a4paper]{article}

\usepackage{fontspec}
\usepackage{polyglossia}
\setdefaultlanguage{thai}
\setotherlanguage{english}

\setmainfont{Sarabun}[
    Script=Thai,
    Language=Default
]

% English font - using Times which is available on macOS
% If you want a different font, you can change this or comment it out
\newfontfamily\englishfont{Times}[
    Ligatures=TeX
]

\usepackage{amsmath}
\usepackage{amsfonts}
\usepackage{amssymb}
\usepackage{graphicx}
\usepackage{hyperref}

% Author names - edit these directly
\newcommand{\authornameone}{Author 1}
\newcommand{\authornametwo}{Author 2}
\newcommand{\authornamethree}{Author 3}
\newcommand{\authornamefour}{Author 4}
\newcommand{\authornamefive}{Author 5}

\title{Transformer: Linear Algebra that understand human language}
\author{%
    \authornameone \\
    \authornametwo \\
    \authornamethree \\
    \authornamefour \\
    \authornamefive
}
\date{\today}

\begin{document}

\maketitle

\begin{abstract}
นี่คือเอกสารตัวอย่างที่รองรับภาษาไทยและภาษาอังกฤษ
This is a sample document that supports both Thai and English.
\end{abstract}

\section{บทนำ}

นี่คือส่วนของบทนำที่เขียนด้วยภาษาไทย

\section{Introduction}

This is an introduction section in English.

\section{ตัวอย่างการเขียนสมการ}

ตัวอย่างสมการทางคณิตศาสตร์:

\begin{equation}
    \mathbf{y} = \text{Transformer}(\mathbf{x})
\end{equation}

\end{document}
